%Declaración del documento%
\documentclass[letterpaper]{article}
%Paquetes%
\usepackage{mathtools}
\usepackage[spanish]{babel}
\usepackage[utf8]{inputenc}
\usepackage{fullpage}

%Datos%
\author{}
\date{}
\title{Integrales de uso frecuente para las series de Fourier}

%Documento%
\begin{document}
\maketitle
\begin{align}
    & \int{sin(nx)dx}=-\frac{1}{n}cos(nx)+C  \\     %1%
    & \int{cos(nx)dx}=\frac{1}{n}sin(nx)+C \\       %2
    & \int{xsin(nx)dx}=\frac{1}{n^2}sin(nx)+\frac{x}{n}cos(nx)+C \\  %3
    & \int{xcos(nx)dx}=\frac{1}{n^2}cos(nx)+\frac{x}{n}sin(nx)+C  \\  %4
    & \int{x^{2}sin(nx)dx}=\frac{2x}{n^2}sin(nx)+\left(\frac{2}{n^3}-\frac{x^2}{n}\right)sin(nx)+C \\  %5
    & \int{x^{2}cos(nx)dx}=\frac{2x}{n^2}cos(nx)+\left(\frac{x^2}{n}-\frac{2}{n^3}\right)sin(nx)+C\\  %6
    & \int{sin(nx)cos(nx)dx} = \frac{1}{2n}sin^2(nx)+C\\ %7
    & \int{sin(mx)sin(nx)dx} = \frac{sin\left[(m-n)x\right]}{2(m-n)}-\frac{sin\left[(m+n)x\right]}{2(m+n)}+C\\ %8
    & \int{sin(mx)cos(nx)dx} = \frac{-cos\left[(m-n)x\right]}{2(m-n)}-\frac{cos\left[(m+n)x\right]}{2(m+n)}+C\\ %9
    & \int{cos(mx)scosin(nx)dx} = \frac{sin\left[(m-n)x\right]}{2(m-n)}+\frac{sin\left[(m+n)x\right]}{2(m+n)}+C %10
\end{align}
\end{document}